\section{Análise dos Resultados}

O tempo do processo sequencial foi de 70 minutos, e os outros tempos estão descritos nas tabelas 1 e 2 em segundos. A versão paralela do algoritmo obteve um desempenho superior a versão sequencial, desempenho que foi aumentando conforme o número de processos cresce. isso se dá porque não há dependência entre os processos, e, pela complexidade do \emph{bubblesort} de O($n^{2}$), ordenar as duas metades do vetor paralelamente faz o trabalho 4 vezes mais rápido idealmente. O desempenho só degrada quando os vetores começam a ficar muito pequenos, e o tempo de dividir o vetor, mandar as mensagens e unir de novo não compensa, pois ordenar seria mais rápido.

O balanceamento de carga se mostrou viável para o algoritmo, que teve um desempenho melhor na sua versão otimizada, pois o balanceamento é mais igualitário. Na versão simples a carga é dividida apenas pelas folhas da árvore, contendo $(n + 1) / 2$ processos, por a árvore de processamento se tratar de uma árvore binária cheia, devido aos números de processos de teste, mas na versão com ordenação local, todos os processos tentam manter um equilíbrio de trabalho.

No problema proposto, é possível utilizar \emph{hyper-threading} com qualidade, pois o trabalho dos processos é bem independente uns dos outros, as únicas situações em que o uso de \emph{hyper-threading} não trouxe muitos ganhos foi no caso de vetores menores, onde o problema era o número de processos, e não o \emph{hyper-threading} em si.

Incluir um percentual para ordenação local permitiu um balanceamento melhor da carga do programa, trazendo um melhor aproveitamento dos processos e uma melhor eficiência do programa. Na versão simples do programa, apenas em torno da metade dos processos estavam realmente trabalhando fazendo a ordenação, enquanto o resto ficava ocioso esperando pelo resultado, utilizando um percentual local todos os processos trabalham na ordenação, minimizando o tempo ocioso ao máximo. A quantia escolhida para ordenação local é o tamanho do vetor dividido pelo número de processos, com um objetivo de todos os processos ordenarem um tamanho igual de vetor. Isso trás dois benefícios, o primeiro é que diminui o tempo ocioso dos processos, pois todos tem uma quantidade igual de trabalho para ordenar, o único trabalho extra será na hora de unir os vetores ordenados, onde, infelizmente, haverá uma disparidade da carga do trabalho e os vetores mais próximos da raiz trabalharão mais. O outro benefício é que igualando o tamanho dos vetores a serem ordenados diminui o tempo de duração do \emph{bottleneck} do programa, que é o \emph{bubblesort}, pois deixa os vetores ordenáveis no menor tamanho possível.