\section{Análise dos Resultados}

\subsection{Análise do balanceamento de carga}

Na versão mais simples do programa, a carga não é igualmente balanceada, sendo todo o trabalho feito nas folhas da árvore, deixando os outros nodos inativos na maior parte do tempo. Com os casos de testes propostos, por contar com árvores de processamento que formam uma árvore binária cheia, apenas (n + 1) / 2 processos estão efetivamente trabalhando, enquanto os outros estarão ociosos.

Na versão com ordenação local, existe uma tentativa de balancear a carga mais igualitáriamente. Em cada nodo não folha, o vetor é dividido em 3 partes iguais, uma para ser ordenada localmente e as outras para serem ordenadas pelos nodos inferiores. Dessa forma a carga fica muito mais bem balanceada que na versão anterior, mas ainda não é completamente igualitário, pois existe o tempo de merge dos vetores ordenados pelos filhos, que é uma carga extra que os processos têm que assumir, que é maior o mais próximo esse processo se encontra da raíz.

\subsection{Análise do ganho obtido com hyperthreading}
Si num tem leite então bota uma pinga aí cumpadi! Não sou faixa preta cumpadi, sou preto inteiris, inteiris. Nec orci ornare consequat. Praesent lacinia ultrices consectetur. Sed non ipsum felis. Diuretics paradis num copo é motivis de denguis.

\subsection{Análise do ganho obtido com a inclusão de um percentual para ordenação local}

Manduma pindureta quium dia nois paga. Mauris nec dolor in eros commodo tempor. Aenean aliquam molestie leo, vitae iaculis nisl. Quem num gosti di mum que vai caçá sua turmis! in elementis mé pra quem é amistosis quis leo.
