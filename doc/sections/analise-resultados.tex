\section{Análise dos Resultados}

O tempo do processo sequencial foi de 70 minutos, os resultados obtidos com a implementação onde o
processamento ocorre pode ser observado nas Tabelas~\ref{table:parallel-100k}
e~\ref{table:parallel-1M}, é possível perceber uma melhora de desempenho em relação ao sequencial, 
isso occorre em função da complexidade do \emph{bubblesort} O$(n^2)$ pois, uma divisão ao meio 
faz o trabalho quatro vezes mais rápido. O desempenho só degrada quando os vetores começam a ficar
muito pequenos, e o tempo de dividir o vetor, mandar as mensagens e unir de novo não compensa, pois
ordenar seria mais rápido.

\begin{table}[h!]
\centering
\caption{Resultados sem ordenação local $10^5$ elementos}
\label{table:parallel-100k}
\begin{tabular}{ccccc}
 \textbf{Processos} & \textbf{Tempo} & \textbf{Speed-Up} & \textbf{Eficiência} \\
 3                & 12.1           & 1,0               & 0,3                 \\
 7                & 4.1            & 2,9               & 0,4                 \\
 15               & 1.9            & 6,4               & 0,4                 \\
 31               & 1.4            & 8,1               & 0,3                
\end{tabular}
\end{table}



\begin{table}[h!]
\centering
\caption{Resultados sem ordenação local $10^6$ elementos}
\label{table:parallel-1M}
\begin{tabular}{ccccc}
\textbf{Processos} & \textbf{Tempo} & \textbf{Speed-Up} & \textbf{Eficiência} \\
 3                & 1.200          & 1,0               & 0,3                 \\
 7                & 303,2          & 4,0               & 0,6                 \\
 15               & 80,7           & 14,9              & 1,0                 \\
 31               & 35,8           & 33,4              & 1,1                 \\
\end{tabular}
\end{table}

%e os outros tempos estão descritos nas tabelas 1 a 4 em segundos. A versão paralela do algoritmo obteve um desempenho superior a versão sequencial, desempenho que foi aumentando conforme o número de processos cresce. isso se dá porque não há dependência entre os processos, e, pela complexidade do \emph{bubblesort} de O($n^{2}$), ordenar as duas metades do vetor paralelamente faz o trabalho 4 vezes mais rápido idealmente. O desempenho só degrada quando os vetores começam a ficar muito pequenos, e o tempo de dividir o vetor, mandar as mensagens e unir de novo não compensa, pois ordenar seria mais rápido.

O balanceamento de carga se mostrou viável para a implementação paralela, que teve um desempenho
melhor na sua versão otimizada, pois o balanceamento é mais igualitário. Na versão simples a 
carga é dividida apenas pelas folhas da árvore, contendo $\frac{n+1}{2}$ processos, pois a árvore de
processamento se tratar de uma árvore binária cheia, devido aos números de processos de teste, mas
na versão com ordenação local, todos os processos tentam manter um equilíbrio de trabalho.

A tentativa de otimização que foi uilizada foi pensada para que nenhum processo ficasse ocioso, com
isso o vetor é dividido igualmente entre os processos, ou seja, para \emph{n} processo cada processo
recebe $\frac{1}{n}$ do tamanho do vetor para ordenar com isso o tempo ocioso diminui . Nas
Tabelas~\ref{table:optimized-100k} e~\ref{table:optimized-1M} estão apresentados os resultadosos
dados a partir desta solução. 

\begin{table}[h!]
\centering
\caption{Resultados com ordenação local $10^5$ elementos}
\label{table:optimized-100k}
\begin{tabular}{ccccc}
 \textbf{Processos} & \textbf{Tempo} & \textbf{Speed-Up} & \textbf{Eficiência} \\
 3                & 5.4            & 1,0               & 0,3                 \\
 7                & 2.0            & 2,6               & 0,4                 \\
 15               & 1.4            & 3,9               & 0,3                 \\
 31               & 1.2            & 4,5               & 0,1                
\end{tabular}
\end{table}

\begin{table}[h!]
\centering
\caption{Resultados com ordenação local $10^6$ elementos}
\label{table:optimized-1M}
\begin{tabular}{ccccc}
 \textbf{Processos} & \textbf{Tempo} & \textbf{Speed-Up} & \textbf{Eficiência} \\
 3                & 502.361595     & 1,0               & 0,3                 \\
 7                & 92.864404      & 5,4               & 0,8                 \\
 15               & 22.246438      & 22,6              & 1,5                 \\
 31               & 10.790653      & 46,6              & 1,5                 \\
\end{tabular}
\end{table}

Incluir um percentual para ordenação local permitiu um balanceamento melhor da carga do programa,
trazendo um melhor aproveitamento dos processos e uma melhor eficiência do programa. Na versão
simples do programa, apenas a da metade dos processos estavam realmente trabalhando fazendo a
ordenação, enquanto o resto ficava ocioso esperando pelo resultado, utilizando um percentual local
todos os processos trabalham na ordenação, minimizando o tempo ocioso. Isso trás dois benefícios, o
primeiro é que diminui o tempo ocioso dos processos, pois todos tem uma quantidade igual de trabalho
para ordenar, o único trabalho extra será na hora de unir os vetores ordenados, onde, infelizmente,
haverá uma disparidade da carga do trabalho e os vetores mais próximos da raiz trabalharão mais. O
outro benefício é que igualando o tamanho dos vetores a serem ordenados diminui o tempo de duração
do \emph{bottleneck} do programa, que é o \emph{bubblesort}, pois deixa os vetores ordenáveis no
menor tamanho possível.

No problema proposto, é possível utilizar \emph{hyper-threading} com qualidade, pois o trabalho dos processos é bem independente uns dos outros, as únicas situações em que o uso de \emph{hyper-threading} não trouxe muitos ganhos foi no caso de vetores menores, onde o problema era o número de processos, e não o \emph{hyper-threading} em si.
