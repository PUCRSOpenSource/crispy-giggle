\section{Modelo da Solução}

Para a solução foram utilizados dois modelos, o primeiro modelo foi implementado de forma que um processo só vai trabalhar na ordenação quando ele for uma folha, caso contrário o vetor é divido ao meio e cada metade é passada a outros dois processos. Com isso podemos identificar duas etapas uma onde a tarefa é delegada para outro processo e outra de ordenação, após ordenar o próximo passo é fazer a união dos vetores menores já ordenados, assim cada processo retorna o resultado ao processo que delegou a tarefa.

A segunda solução utiliza ordenação local, então, em vez de dividir o vetor em dois, cada processo divide ele em três, guardando uma parte para ordenar localmente, com o \emph{bubblesort}. Isso trás um ganho de eficiência, pois diminui o tempo que os processos ficam parados, que será analisado posteriormente. A quantidade escolhida para ordenação total é o tamanho total do vetor dividido pelo número de processos, para balancear a carga da melhor maneira.