\section{Modelo da Solução}

Foram modeladas duas soluções diferentes, uma com ordenação local e a outra sem. Na solução sem ordenação local, o primeiro processo vai pegar o vetor inteiro desordenado, dividir em dois, e passar adiante para que os próximos processos o ordenem, quando ordenado ele recebe de volta e faz o merge dos dois vetores que recebeu ordenado. Os processos vão delegando a função de ordenar o vetor pra baixo até não haver mais processos para passar adiante, nas folhas da árvore. Chegando nas folhas, o processo executa o bubblesort sobre o vetor e envia ele de volta para cima ordenado.

A segunda solução utiliza ordenação local, então, em vez de dividir o vetor em dois, cada processo divide ele em três, guardando uma parte para ordenar localmente, com o bubblesort. Isso trás um ganho de eficiência, pois diminui o tempo que os processos ficam parados, que será analisado posteriormente.